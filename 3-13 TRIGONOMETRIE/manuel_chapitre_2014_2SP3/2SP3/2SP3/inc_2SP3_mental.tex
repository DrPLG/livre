\serie{Activités mentales}

\begin{exercice*}
 On considère un dé pipé.\\ En utilisant le tableau suivant, 
 calculer $p(6)$.
 
 \begin{tableau}[C]{\linewidth}{7}{m{2cm}}
  \hline
  Face & 1&2&3&4&5&6\\\hline
  Probabilité &0,1&0,2&0,1&0,15&0,25&\\\hline
 \end{tableau}
 \begin{corrige}
  $p(6)=0,2$
 \end{corrige}
\end{exercice*}

\begin{exercice*}
 On considère deux événements $A$ et $B$ tels que:
 \begin{colitemize}{2} 
 \item $p(A)=0,6$
  \item $p(B)=0,5$
   \end{colitemize}
   \vspace{-1.5em}
   \begin{itemize}   
  \item $p(A\cap B)=0,3$ 
   \end{itemize}
    \vspace{-1.5em}
 Calculer $p(A\cup B)$
 \begin{corrige}
  $p(A\cup B)=0,8$
 \end{corrige}
\end{exercice*}

\begin{exercice*}
 On considère deux événements $A$ et $B$ tels que:
 \begin{colitemize}{2}
 \item $p(A)=0,7$;  \item $p(B)=0,5$ 
 \item $p(A\cup B)=0,9$
 \end{colitemize}
 Calculer $p(A\cap B)$.
 \begin{corrige}
  $p(A\cap B)=0,3$
 \end{corrige}
\end{exercice*}



\begin{exercice*}
 On considère deux événements $A$ et $B$ tels que:
 \begin{colitemize}{2}
 \item $p(A)=0,5$
 \item $p(B)=0,8$
 \item $p(A\cap B)=0,4$ 
 \end{colitemize}
 Calculer $p\left(\overline{A\cup B}\right)$
 \begin{corrige}
 $p\left(\overline{A\cup B}\right)= 0,1$
 \end{corrige}
\end{exercice*}

\begin{exercice*}
 On choisit au hasard un des 7 nains. 
 Quelle est la probabilité que ce soit Joyeux ou Atchoum ?
 %\vspace{-1.5em}
 \begin{corrige}
  $2/7$
 \end{corrige}
\end{exercice*}

\begin{exercice*}
 On tire au hasard une pièce d'un échiquier.\\ Soit $C$ l'événement: \og la pièce est une tour ou elle est blanche\fg .\\ 
 Exprimer $\overline{C}$ par une phrase.
 \begin{corrige}
  \og la pièce est noire et n'est pas une tour.
 \end{corrige}
\end{exercice*}

\begin{exercice*}
 Une famille a deux enfants. Quelle est la probabilité qu'ils soient de sexes différents?
 \begin{corrige}
  0,5
 \end{corrige}
\end{exercice*}

\begin{exercice*}
 $A$ et $B$ sont deux événements incompatibles.
 \begin{enumerate}
 \item  Calculer $p(A\cap B)$ si:
 \begin{colitemize}{2} \item $p(A)=0,3$  \item $p(B)=0,5$ \end{colitemize} 
\item  Calculer $p(A\cup B)$ si:
\begin{colitemize}{2} \item $p(A)=0,2$ \item $p(B)=0,4$\end{colitemize}
\end{enumerate}
 \begin{corrige}
 ~\\
 \begin{enumerate}
 \item $p(A\cap B)=0$
 \item $p(A\cup B)=0,6$
 \end{enumerate}
   \end{corrige}
\end{exercice*}

\begin{exercice*}
Une mère a deux garçons et attend un 3\ieme\ enfant. 

Quelle est la probabilité pour que ce soit une fille ?
 \begin{corrige}
  0,5
 \end{corrige}
\end{exercice*}

\begin{exercice*}
On choisit un élève au hasard dans une classe. Quel est l'événement
contraire de l'événement: 
\begin{enumerate}
\item \og c'est une fille qui a appris sa leçon\fg ?
\item \og c'est une fille ou un élève qui appris sa
leçon\fg ?
\end{enumerate}
 \begin{corrige}
 ~\\
 \begin{enumerate}
 \item \og C'est un garçon ou un élève qui n'a pas appris sa leçon \fg
 \item \og C'est un garçon qui n'a pas appris sa leçon \fg
 \end{enumerate}  
 \end{corrige}
\end{exercice*}

\begin{exercice*}
Un élève répond au hasard à un Q.C.M. comportant 5 questions. Quel est
l'événement contraire de:\\ \og il a répondu juste à au moins 2
questions \fg ?
 \begin{corrige}
  \og répondre juste à moins deux questions.\fg
 \end{corrige}
\end{exercice*}



