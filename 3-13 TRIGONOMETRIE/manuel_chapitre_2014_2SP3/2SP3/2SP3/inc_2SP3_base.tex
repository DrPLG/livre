%5 pages
\serie{Expérience aléatoire}
\begin{exercice}\label{2SP3_E_E1}
     On lance cinq fois une pièce de monnaie. La sortie de Pile rapporte 1 
 point. La sortie de Face ne rapporte rien. On s'intéresse à la somme des
 points obtenus à l'issue des cinq lancers.
 \begin{enumerate}
 \item Décrire l'univers associé à l'expérience aléatoire.
 \item Préciser le nombre d'éventualités qui le composent.
 \end{enumerate}
 \end{exercice}
 %3
 \begin{exercice}Même consigne que l'exercice \RefExercice{2SP3_E_E1}.\\
     On lance deux dés cubiques dont les faces sont numérotées de 1 à 6 et on 
 soustrait le plus petit résultat obtenu du plus grand. Le résultat est
 nul si le lancer produit un double.
 \end{exercice}
 %4
 \begin{exercice} Même consigne que l'exercice \RefExercice{2SP3_E_E1}.\\
 Pour se rendre du point $A$ au point $B$, on choisit au hasard un trajet 
 parmi tous les trajets possibles en se déplaçant d'un "pas" à droite ou
 un "pas" vers le haut.
     \begin{center}    
     \begin{tikzpicture}[general]
     \draw [quadrillage] 
 (0,0) grid (2,2);
     \clip(-1,-1) rectangle (3,3);
     \fill [color=A1] (0,0) circle (1.5pt);
     \draw[color=A1] (0.03,0.06)[above left] node {$A$};
     \fill [color=A1] (2,2) circle (1.5pt);
     \draw[color=A1] (1.98,1.95) [above right] node {$B$};
     \end{tikzpicture}
     \end{center}
 \end{exercice}
 %5
 \begin{exercice}
     \`A partir du lancer simultané de deux dés tétraédriques, imaginer cinq 
 expériences aléatoires conduisant à cinq univers différents.
 \end{exercice}
 %6
 \begin{exercice}
 On lance deux dés à trois faces équilibrés et on s'intéresse à la somme des 
 nombres obtenus. On définit les événements suivants :
     \begin{itemize}
         \item $E$ : \og le résultat est pair \fg 
         \item $F$ : \og le résultat est au moins égal à 5 \fg
         \item $G$ : \og le résultat est au moins égal à 6 \fg
     \end{itemize}
 Préciser les événements élémentaires qui composent chacun des événements ci 
 dessus.
 \end{exercice}
 %7
 \begin{exercice}
     Une personne pressée répond au hasard à un sondage. Deux questions sont 
 posées et à chacune, on donne le choix entre:
 \begin{colitemize}{2}
 \item favorable
 \item opposé
 \item sans opinion
 \end{colitemize}
     De combien de façons la personne peut répondre?
 \end{exercice}
 
\serie {Vocabulaire des événements}

%modifié par benoit
\begin{exercice}
On tire une carte d'un jeu de 32~cartes. \\On appelle:
\begin{itemize}
\item $C$ l'événement \og la carte tirée est un c\oe ur\fg
\item $F$ l'événement \og la carte tirée est une figure\fg 
\end{itemize}
\vspace{-1.5em}
\begin{enumerate}
 \item Décrire par une phrase l'événement $C\cap F$.

 Combien compte-t-il d'issues?
 
 \item Décrire par une phrase l'événement $C\cup F$.
 
 Combien compte-t-il d'issues?
 \item  Décrire par une phrase l'événement $\overline{C}\cap F$. 
 
 Combien compte-t-il d'issues?
  \item  Décrire par une phrase l'événement $\overline{C\cup F}$. 
 
 Combien compte-t-il d'issues?
\end{enumerate}
\end{exercice}

%modification de Benoit : suppression de l'espace avant le begin enumerate
\begin{exercice}
 Deux épidémies sévissent en même temps dans un lycée, la gastro-entérite et un rhume. On choisit un élève au hasard et on nomme:
 \begin{itemize}
 \item $G$ l'événement \og l'élève a la gastro-entérite\fg{}
 \item  $R$ l'événement \og l'élève a un rhume\fg 
\end{itemize} 
 Décrire à l'aide de ces deux événements: 
 \begin{enumerate}
  \item \og l'élève a la gastro-entérite et le rhume\fg 
  \item \og l'élève a le rhume mais pas la gastro-entérite\fg 
  \item \og l'élève a au moins une des deux maladies\fg 
  \item \og l'élève n'a aucune des deux maladies\fg 
 \end{enumerate}

\end{exercice}

%1
\begin{exercice}\label{horloge numérique}
On regarde à un instant au hasard l'heure affichée par une 
horloge à affichage numérique, et on note le chiffre des dizaines du
nombres des
    minutes.
    \begin{enumerate}
    \item Décrire l'univers associé à l'expérience aléatoire.
    \item Préciser le nombre d'éventualités qui le composent.
    \end{enumerate}
\end{exercice}

%8
\begin{exercice}
    Dans une classe de 13 filles et 16 garçons, on désigne au hasard une fille et un garçon pour être délégués
    provisoires.\\
    Combien y a-t-il de couples possibles?
\end{exercice}

%Modification de Benoit : rajout du colitemize qui avait disparu
\begin{exercice} Construire un diagramme de Venn (sur le modèle ci-dessous) pour chacun des événements suivants.
  \begin{colitemize}{3}
  	\item $A\cap \overline{B}$
  	\item $A\cup \overline{B}$  	
  	\item $\overline{A\cap B}$
  	\item $\overline{A\cup B}$  	
  	\item $\overline{A}\cap \overline{B}$
  	\item $\overline{A}\cup \overline{B}$
  \end{colitemize}
  
 \begin{tikzpicture}[general, yscale=0.4]
\draw (-4,-1) rectangle (4,4);% E
\draw[color=A1] (0,0) ++(135:2) circle (2); % A
\draw[color=F1] (0,0) ++(45:2) circle (2); % B
\draw (-2,1.5) node {$A$};
\draw (2,1.5) node {$B$};
\draw (3,3.5) node {$\Omega$};
\end{tikzpicture}

\end{exercice}

\serie{Calcul de probabilités }
\begin{exercice}
 Soit $A$ et $B$ deux événements tels que:
 \begin{colitemize}{2}
 \item $p(A)=0,7$
 \item $p(B)=0,5$
 \item $p(A\cap B)=0,3$
\end{colitemize} 
En s'aidant d'un diagramme de Venn, calculer:
\begin{colenumerate}{3}
\item $p\left(\overline{A}\right)$
\item $p(A\cup B)$
\item $p\left(\overline{A}\cap B\right)$
\end{colenumerate} 
\end{exercice}

\begin{exercice}
 Soit $S$ et $T$ deux événements tels que:
 \begin{colitemize}{2}
 \item $p(S)=0,5$
 \item $p(T)=0,6$
 \item $p(S\cup T)=0,9$
 \end{colitemize}
 Calculer les probabilités suivantes:
 \begin{colenumerate}{3}
 \item $p(S\cap T)$
 \item $p\left(\overline{S\cup T}\right)$
 \item $p\left(\overline{S\cap T}\right)$.
 \end{colenumerate}
\end{exercice}

\begin{exercice}
 Robin des Bois atteint la cible avec une probabilité de 0,7. Quelle est la probabilité qu'il rate sa cible?
\end{exercice}

\begin{exercice}
 $A$ et $B$ sont deux événements incompatibles.
 \begin{colitemize}{2}
 \item $p(A)=0,4$
 \item $p(B)=0,22$
 \end{colitemize} 
 Déterminer la probabilité des événements suivants: 
 \begin{colenumerate}{3}
  \item $\overline{A}$
  \item $\overline{B}$
   \item $A\cup B$
   \end{colenumerate}
\end{exercice}

\begin{exercice}
 $A$ et $B$ sont deux événements tels que:
 \begin{colitemize}{2}
 \item $p(A)=0,8$ \item $p(B)=0,53$
 \end{colitemize}
 \vspace{-1.5em}
  \begin{enumerate}
  \item $A$ et $B$ sont-ils incompatibles?
  \item Sachant que $p(A\cup B)=0,95$, calculer:
  \begin{colenumerate}{2} \item $p(A\cap B)$ \item $p\left( A \cap \overline{B}\right)$ \end{colenumerate}
  \vspace{-1.5em}
 \end{enumerate}
\end{exercice}

\begin{exercice}
 On considère 2 événements $V$ et $F$ tels que:
 \begin{colitemize}{2}
 \item $p(V)=0,4$ \item $p(F)=0,3$ \item $p(V\cup F)=0,8$ 
 \end{colitemize}
 Aïssatou prétend que ce n'est pas possible. \\Confirmer ou infirmer sa déclaration. 
\end{exercice}

%\begin{exercice}
% On considère 2 événements $V$ et $F$ tels que:
% \begin{colitemize}{2}
% \item $p(V)=0,6$;
% \item $p(F)=0,4$;
% \item $p(V\cup F)=0,8$. 
% \end{colitemize} 
%Aminata prétend que ce n'est pas possible.\\ Confirmer ou infirmer sa déclaration. 
%\end{exercice}

\begin{exercice}
 On considère 2 événements $V$ et $F$ tels que:
 \begin{colitemize}{2}
 \item $p(V)=0,6$ \item $p(F)=0,4$ \item $p(V\cap F)=0,5$ 
 \end{colitemize} 
Arinucea prétend que ce n'est pas possible.\\ Confirmer ou infirmer sa déclaration. 
\end{exercice}

\begin{exercice}
 On considère 2 événements $V$ et $F$ tels que:
 \begin{colitemize}{2} \item $p(V)=0,6$ \item $p(F)=0,4$ \item $p(V\cap F)=0,4$ \end{colitemize}
Ataroa prétend que ce n'est pas possible.\\ Confirmer ou infirmer sa déclaration. 
\end{exercice}

% \begin{exercice}
%  On lance un dé pipé tel que 
%  
%  $p(1)=2p(2)=3p(3)=4p(4)=5p(5)=6p(6)$
%  
%  Calculer $p(1)$.
% \end{exercice}

\begin{exercice}
 On considère deux événements $V$ et $F$ tels que $p(V)=0,6$ et $p(V\cup F)=0,7$. Ahuarii prétend que ce n'est pas possible. Confirmer ou infirmer sa déclaration. 
\end{exercice}


\serie{\'Equiprobabilité}

\begin{exercice*}\ExerciceRefMethode{2SP3_M_équiprobabilité}\\\label{2SP3_E_équiprobabilité}
On lance 3 fois une pièce bien équilibrée.
\begin{enumerate}
\item Représenter la situation par un arbre.
\item Quelle est la probabilité: 
\begin{enumerate}
\item d'avoir 3 faces?
\item que le 2\ieme\ jet soit face?
\item que le 3\ieme\ jet soit différent du 1\ier? 
\end{enumerate}
\end{enumerate}
\begin{corrige}
\begin{enumerate}
\setcounter{enumi}{1}
\item \begin{colenumerate}{2}
\item $1/8$
\item $1/2$
\item $1/2$
\end{colenumerate}
\end{enumerate}
\end{corrige}
\end{exercice*}

\begin{exercice}
Deux dés tétraédriques ont des faces numérotées de 1 à 4. On les lance et on regarde la somme obtenue.
\begin{enumerate}
\item Quels sont les résultats possibles?
\item Est-ce une situation d'équiprobabilité?
\item Déterminer la probabilité de chaque résultat.
\end{enumerate}
\end{exercice}

\begin{exercice}
      On lance un dé bien équilibré à six faces dont trois sont bleues, deux sont 
 blanches et une est rouge.
     \begin{enumerate}
         \item Les trois couleurs sont-elles équiprobables?
         \item Déterminer la probabilité d'apparition de chaque couleur.
     \end{enumerate}
 \end{exercice}
 \begin{exercice}[Dé à 4 faces]
     On lance deux dés à quatre faces et on regarde la somme obtenue.
     \begin{enumerate}
         \item Donner l'ensemble des résultats possibles.
         \item Donner une loi de probabilités de cette expérience 
 aléatoire (justifier).
         \item Quelle est la probabilité d'obtenir un nombre pair?
         \item Quelle est la probabilité d'obtenir un nombre multiple 
 de trois?
     \end{enumerate}
 \end{exercice}

\begin{exercice}[Menus]
Au restaurant scolaire, les élèves ont le choix 
\begin{itemize}
\item entre 2 entrées: Artichaut ou Betterave;
\item entre 3 plats: Cheval, Daube ou Escalope;
\item entre 2 desserts: Fromage ou Gâteau.
\end{itemize}
Un menu se compose:
\begin{colitemize}{3}
\item d'une entrée; \item d'un plat;  \item d'un dessert.
\end{colitemize}
\vspace{-1.5em}
\begin{enumerate}
\item En utilisant un arbre, représenter tous les menus.
\item Combien de menus différents sont possibles?
\item On choisit un menu au hasard. \\ Quelle est la probabilité:
\begin{enumerate}
\item qu'il comporte une escalope?
\item qu'il comporte de l'artichaut et du fromage?
\item qu'il ne comporte pas de cheval?
\end{enumerate}
\end{enumerate}
\end{exercice}

%%%%%%%%%%%%%29
\begin{exercice}
Un groupe de 4 amis, \'Emile, Flore, Gaston et Hélène sont dans un bateau. Ils tirent au sort celui qui va ramer et, parmi les noms restants, celui qui va écoper.
\begin{enumerate}
\item Représenter cette situation par un arbre.
\item Déterminer les probabilités suivantes.
\begin{enumerate}
\item C'est un garçon qui rame.
\item Hélène écope.
\item Les deux qui travaillent sont de même sexe.
\end{enumerate}
\end{enumerate}  
\end{exercice}

%%%%%%%%%%%%%%30


\begin{exercice}[Tirage successif avec remise]
On tire au hasard une carte d'un jeu de 32 cartes, on la note, puis on la remet dans le jeu avant d'en tirer une seconde.
	\begin{enumerate}
		\item Est-ce une situation d'équiprobabilité?
		\item Combien y a-t-il d'issues?
		\item Calculer la probabilité de:
		\begin{enumerate}
			\item tirer 2 c\oe urs;
			\item  ne pas tirer de c\oe ur;
			\item tirer exactement 1 c\oe ur;
			\item tirer deux fois la même carte;
			\item tirer deux cartes différentes;
			\item tirer le roi de c\oe ur.
		\end{enumerate}	
	\end{enumerate}
\end{exercice}

\begin{exercice}[Tirage successif sans remise]
On tire au hasard deux cartes d'un jeu de 32 cartes, l'une après l'autre.
	\begin{enumerate}
		\item Est-ce une situation d'équiprobabilité?
		\item Combien y a-t-il d'issues?
		
		\item Calculer la probabilité de tirer
		\begin{enumerate}
		\item deux c\oe urs;
			\item  exactement 1 c\oe ur;
			\item  deux fois la même carte;
			\item  deux cartes différentes;
			\item  le roi de c\oe ur.
		\end{enumerate}	
		\item  Calculer la probabilité de ne pas tirer de c\oe ur.
	\end{enumerate}
\end{exercice}

\begin{exercice}
Trois CD notés $a$, $b$ et $c$ ont respectivement des boîtes nommées $A$, $B$ et $C$.
On range les 3 CD au hasard dans les boîtes sans voir leur étiquette.
\begin{enumerate}
\item Combien de rangements sont possibles?
\item Quelle est la probabilité 
\begin{enumerate}
\item que les 3 CD soient bien rangés?
\item qu'exactement 1 CD soit bien rangé?
\item qu'exactement 2 CD soient bien rangés?
\end{enumerate}
\item En déduire la probabilité qu'aucun CD ne soit bien rangé.
\end{enumerate}
\end{exercice}

%\begin{exercice}
%On lance deux dés tétraédriques dont les faces sont numérotées de 1 à 4 et on regarde la somme obtenue.
%\begin{enumerate}
%\item Quels sont les résultats possibles?
%\item Déterminer la probabilité de chaque résultat.
%\item Ces résultats sont-ils équiprobables?
%\end{enumerate}
%\end{exercice}





\serie{Sans équiprobabilité}
\begin{exercice} Un univers associé à une expérience aléatoire est constitué de trois issues. La loi de probabilité vérifie $p(A)=t^2$, $p(B)=t$ et $p(C)=\dfrac{\TopStrut 1}{\BotStrut 4}$. Déterminer $t$.
\end{exercice}

\begin{exercice}[Loi de probabilité]
 Pour chacun des cas suivants, déterminer la valeur de $t$ qui permet de définir 
 une loi de probabilités.
     \begin{enumerate}
         \item \begin{center}
                 % use packages: array
                     \begin{tableau}[C]{\linewidth}{3}{m{1.7cm}}
                         \hline
                         Issues & P & F  \\ 
                         \hline
                         Probabilités & 0,3 & $t$ \\
                         \hline
                     \end{tableau}
                 \end{center}
  
 
         \item \begin{center}
                 % use packages: array
                      \begin{tableau}[C]{\linewidth}{4} {m{1.7cm}}       
                         \hline
                         Issues & Vert & Orange & Rouge   \\ 
                         \hline
                         Probabilités & 0,5 & $t$ & 0,3  \\
                         \hline
                     \end{tableau}
                 \end{center}
         
  
         \item \begin{center}
                 % use packages: array
                     \begin{tableau}[C]{\linewidth}{5}{m{1.7cm}}   
                         \hline
                         Issues & 1 & 2 & 3 & 4  \\ 
                         \hline
                         Probabilités & 0,25 & $t$ & 0,2 & 0,4  \\
                         \hline
                     \end{tableau}
                 \end{center}
 
         \item \begin{center}
                 % use packages: array
                     \begin{tableau}[C]{\linewidth}{7}{m{1.7cm}}   
                         \hline
                         issues & 1 & 2 & 3 & 4 &5 &6 \\ 
                         \hline
                         probabilités & $t$ & $2t$ & $3t$ & $4t$ &$5t$ &$6t$ \\
                         \hline
                     \end{tableau}
                 \end{center}
 
     \end{enumerate}
 \end{exercice}
\begin{exercice}[Au tennis]
    Un joueur de tennis de niveau international a une probabilité 0,58 de 
réussir son premier service et une probabilité de 0,06 de faire une double
    faute.\\
    Quelle est la probabilité qu'il réussisse seulement son deuxième service?
\end{exercice}

\begin{exercice}
Une entreprise fabrique des ordinateurs portables. Ils peuvent présenter deux défauts:
\begin{colitemize}{2}
\item un défaut de clavier;
\item un défaut d'écran.
\end{colitemize}
 Sur un grand nombre d'ordinateurs, une étude statistique montre que:
\begin{itemize}
\item \upc{2} présentent un défaut d'écran;
\item \upc{2,4} présentent un défaut de clavier;
\item \upc{1,5} présentent les deux défauts.
\end{itemize}
\vspace{-0.75em}
\begin{enumerate}
\item On choisit au hasard un ordinateur et on considère les événements suivants.
\begin {itemize}
\item $E$: \og L'ordinateur présente un défaut d'écran \fg{};
\item $C$: \og L'ordinateur présente un défaut de clavier \fg{}.
\end{itemize} Détermine $p(E)$, $p(C)$ et $p(E \cap C)$.
\item On considère les événements suivants.
\begin{itemize}
\item \og L'ordinateur présente au moins un défaut \fg{};
\item \og L'ordinateur ne présente que le défaut de d'écran \fg{}.
\end{itemize}
\vspace{-1.5em}
\begin{enumerate}
\item Traduire ces 2  événements à l'aide de $E$ et $C$.
\item Calcule leur probabilité.
\end{enumerate}
\end{enumerate}
\end{exercice}



\begin{exercice*}\ExerciceRefMethode{2SP3_M_équiprobabilité}\label{2SP3_E_équiprobabilité_2}\\
On lance un dé pipé. \\Le tableau suivant regroupe les probabilités.
\begin{tableau}[lc]{\linewidth}{7}\hline
$F$&1&2&3&4&5&6\\\hline
$p(F)$&0,1&0,1&0,2&0,2&0,3&?\\\hline
\end{tableau}
\begin{enumerate}
\item Calculer $p(6)$.
\item Calculer la probabilités des événements suivants.
\begin{enumerate}
\item \og La face obtenue est paire \fg{};
\item \og la face obtenue est supérieur ou égale à 5 \fg{}. 
\end{enumerate}
\vspace{-1.5em}
\end{enumerate}
\begin{corrige}
\begin{enumerate}
\item 0,1
\item \begin{colenumerate}{2} \item 0,4 \item 0,4 \end{colenumerate}
\end{enumerate}
\end{corrige}
\end{exercice*}

\begin{exercice}[S'arrêter]
Voici le cycle d'allumage d'un feu tricolore:
\begin{itemize}
\item \us{45} pour le feu vert;
\item \us{5} pour le feu orange;
\item \us{20} pour le feu rouge.
\end{itemize}
En admettant qu'un automobiliste arrive par hasard devant un feu tricolore fonctionnel, déterminer la loi de probabilité associée à cette expérience.
\end{exercice}

\begin{exercice}[Prendre rendez-vous!]
Le standard d'un cabinet médical dispose de deux lignes de téléphone. On considère les événements:
\begin{itemize}
\item$O_1$: \og La 1\ier\ ligne est occupée \fg.
\item $O_2$: \og La 2\ieme\ ligne est occupée \fg. 
\end{itemize}
Une étude statistique montre que:
\begin{itemize}
\item $p(O_1)=0,4$
\item $p(O_2)=0,3$
\item $p(O_1 \cap O_2)=0,2$
\end{itemize}
Calculer la probabilité des événements suivants.
\begin{enumerate}
\item \og La ligne 1 est libre \fg.
\item \og Au moins une des lignes est occupée \fg.
\item \og Au moins une des lignes est libre \fg. 
\end{enumerate}
\end{exercice}

\begin{exercice}
    Un magazine de jeux vidéos souhaite étudier les ventes des deux consoles 
de next-gen:
\begin{itemize}
\item celle de Megahard : la Z-boite 2;
\item celle de Silency: la Gare de jeux.
\end{itemize}
Pour cela elle étudie les ventes sur une journée dans un magasin de 
centre-ville. Elle constate que sur les 127 ventes de consoles next-gen de la journée, 53 
personnes ont acheté la Z-boite 2.
Pedro entre dans le magasin pour acheter une des deux consoles.
\begin{enumerate}
\item Quelle est la probabilité qu'il achète la Gare de jeux?
\item Bill  pense que Pedro à \upc{50} de chances d'acheter chacune des deux consoles.
    Bill a-t-il raison? Pourquoi?
\end{enumerate} 
\end{exercice}

 
 

 
 
 
 
 
 %23
 
 





