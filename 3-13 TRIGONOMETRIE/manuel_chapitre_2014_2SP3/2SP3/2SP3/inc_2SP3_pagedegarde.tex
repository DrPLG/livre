\begin{prerequis}
 \begin{itemize}
\item Calculer et utiliser des fréquences 
\item Calculer et utiliser des pourcentages
 \end{itemize}
\end{prerequis}
 
 \begin{autoeval}
  \begin{multicols}{2}
    \begin{exercice}
    Le tableau ci-dessous présente le nombre de pots de peinture vendus en un mois selon la couleur.
    \begin{center}
    \begin{tableau}[LC]{\linewidth}{4}{m{2cm}}\hline
    Couleur&Jaune&Blanc&Rouge\\\hline
	Effectif&256&7489&458\\\hline
    \end{tableau}
    \end{center}
    \begin{center}
    \begin{tableau}[LC]{\linewidth}{4}{m{2cm}}\hline
    Couleur&Bleu&Vert&Noir\\\hline
	Effectif&156&785&4123\\\hline
    \end{tableau}
    \end{center}
    \begin{enumerate}
    \item Calculer les fréquences arrondies au centième.
    \item Exprimer les fréquences en pourcentage arrondies à l'unité.
    \end{enumerate}
    \begin{corrige}
    \begin{center}
    \begin{tableau}[LC]{\linewidth}{4}{m{1.7cm}}
    Couleur&Jaune&Blanc\\\hline
	Fréquences&0,02&0,56\\\hline
	Pourcentage&2&56\\\hline
    \end{tableau}
    \end{center}
    \begin{center}
    \begin{tableau}[LC]{\linewidth}{4}{m{1.7cm}}
    Couleur&Rouge&Bleu\\\hline
	Fréquences&0,03&0,01\\\hline
	Pourcentage&3&1\\\hline
    \end{tableau}
    \begin{tableau}[LC]{\linewidth}{3}{m{1.7cm}}
    Couleur&Vert&Noir\\\hline
	fréquences&0,06&0,31\\\hline
	pourcentage&6&31\\\hline
    \end{tableau}
    \end{center}
    \end{corrige}
 	\end{exercice}
 	\begin{exercice}
 	Dans une boulangerie, Mariette achète:
 	\begin{colitemize}{2} \item 15 pains au chocolat; \item 12 tartelettes; \item 22 éclairs; \item 10 croissants; \item 8 pains au raisin; \item 20 brioches.\end{colitemize}
 	\vspace{-1.5em}
 	\begin{enumerate}
 	\item Quelle est la proportion de: 
 	\begin{colenumerate}{2}
 	\item tartelettes?
 	\item viennoiserie?
 	\end{colenumerate}
 	\item Parmi les desserts, quelle est la proportion d'éclairs?
 	\end{enumerate}
 	\begin{corrige}
 	\begin{enumerate}
 	\item \begin{colenumerate}{2}
 	\item $12/87$
 	\item $53/87$
 	\end{colenumerate}
 	\item $22/34$
 	\end{enumerate}
 	\end{corrige}
 	\end{exercice}
 	\begin{exercice}
 	En 2013, \u{778200} candidats se sont présentés à la série générale de l'examen du Diplôme National du Brevet, \upc{84.5} ont été reçu et neuf candidats sur 10 maîtrisaient le socle commun de compétences.
 	\begin{enumerate}
 	\item Combien de candidats ont été reçus?
 	\item Combien de candidats ont la maîtrise du socle commun de compétences?
 	\end{enumerate}
 	\begin{corrige}
 	\begin{colenumerate}{2}
 	\item \nombre{657579}
 	\item \nombre{700380}
 	\end{colenumerate}
 	\end{corrige}
 	\end{exercice}
 	\begin{exercice}
 	Dans la liste des nombres entiers de 0 à 20, citer 
 	\begin{enumerate}
 	\item les nombres impairs;
 	\item les nombres divisibles par 3;
 	\item les nombres impairs ou divisibles par 3;
 	\item les nombres impairs non nuls et divisibles par 3.
 	\end{enumerate} 	
 	\begin{corrige}
 	~\\
 	\begin{enumerate}
 	\item 1, 3, 5, 7, 9, 11, 13, 15, 17, 19.
 	\item 0, 3, 6, 9, 12, 15, 18.
 	\item 0, 1, 3, 5, 6, 7, 9, 11, 12, 13, 15, 17, 18, 19.
 	\item 0, 3, 9, 15. 
 	\end{enumerate}
 	\end{corrige}
 	\end{exercice}
 	\begin{exercice}
 	Benoît a réparé 351 machines à laver. Il a changé le joint sur 128 machines et le programmateur sur les autres dont 26 présentaient aussi un défaut de joint qu'il a aussi remplacé.
 	\begin{enumerate}
 	\item Quel est le pourcentage de machines à laver ayant un joint défectueux?
 	\item Quel est le nombre de machine ayant seulement un programmateur défectueux?
 	\end{enumerate} 	 	
 	\begin{corrige}
~\\
\begin{enumerate}
\item  \upc{43,9}
\item 197
\end{enumerate}
 	\end{corrige}
 	
 	\end{exercice}
\end{multicols}
 \end{autoeval}